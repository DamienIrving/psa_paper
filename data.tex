\section{Data}

Data from the European Centre for Medium-Range Weather Forecasts Interim Reanalysis \citep[ERA-Interim;][]{Dee2011} were used in this study. In particular, the six-hourly 500 hPa zonal and meridional wind, surface air temperature, sea ice fraction, sea surface temperature and mean sea level pressure analysis fields were used, from which daily mean time series were calculated for the 36-year period 1 January 1979 to 31 December 2014. For precipitation, the `total precipitation' forecast fields were used. Each forecast field represents the accumulated precipitation since initialization, so the daily rainfall total was calculated as the sum of the two 12 hours post initialization accumulation fields for each day. The horizontal resolution of the ERA-Interim data was 0.75$^{\circ}$ latitude by 0.75$^{\circ}$ longitude.  

Relative to the other latest generation reanalysis datasets, ERA-Interim is thought to best reproduce the precipitation variability \citep{Bromwich2011,Nicolas2011}, vertical temperature structure \citep{Screen2012} and mean sea level pressure and 500 hPa geopotential height at station locations \citep{Bracegirdle2012} around Antarctica. While these are encouraging results, it is worth noting that the sparsity of observational data in the mid-to-high southern latitudes means that ERA-Interim (like all reanalysis data) still needs to be interpreted with caution. There are also well-known difficulties with the representation of low-frequency variability and trends in reanalysis data, due to factors such as changes in the observing system over time \citep{Dee2014}. These issues are highly relevant to the PSA pattern trends discussed in this study, but are somewhat less critical for the results pertaining to seasonal and interannual variability.
