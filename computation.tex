\section{Computation}

The results in this paper were obtained using a number of different software packages. A collection of command line utilities known as the NetCDF Operators (NCO) and Climate Data Operators (CDO) were used to edit the attributes of netCDF files and to perform routine calculations on those files (e.g. the calculation of anomalies and climatologies) respectively. For more complex analysis and visualization, a Python distribution called Anaconda was used. In addition to the Numerical Python \citep[NumPy;][]{VanDerWalt2011} and Scientific Python (SciPy) libraries that come installed by default with Anaconda, a Python library called xray was used for reading/writing netCDF files and data analysis. Similarly, in addition to Matplotlib \citep[the default Python plotting library;][]{Hunter2007}, Iris and Cartopy were used to generate many of the figures. Iris was also used for rotating the global coordinate system and meridional wind (via the PROJ.4 Cartographic Projections Library), and the eofs and windspharm libraries were used for EOF analysis and for calculating the streamfunction respectively.

To facilitate the reproducibility of the results presented, an accompanying Figshare repository has been created to document the computational methodology (CREATE FIGSHARE PAGE). In addition to a more detailed account (i.e. version numbers, release dates, web addresses) of the software packages discussed above, the Figshare repository contains a supplementary file for each figure in the paper, outlining the computational steps performed from initial download of the ERA-Interim data through to the final generation of the plot. A version controlled repository of the code referred to in those supplementary files can be found at \url{https://github.com/DamienIrving/climate-analysis}. The rationale behind this approach to documenting computational results is explained by \citet{Irving2015}.