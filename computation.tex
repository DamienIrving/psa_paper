\section{Computation}

A number of different software packages were used in generating the key results presented in this paper. Simple editing of netCDF file attributes and routine data analysis tasks (e.g. anomalies, running means) were performed using a collection of command line utilities known as the NetCDF Operators (NCO) and Climate Data Operators (CDO) respectively, while a Python distribution called Anaconda was used for more complex analysis and visualization. With respect to specific Python libraries, xray was used for data analysis and reading/writing netCDF files, which is a library that builds upon the Numerical Python \citep[NumPy;][]{VanDerWalt2011}, Pandas and Scientific Python (SciPy) libraries that come installed by default with Anaconda. Similarly, Iris, Cartopy and Seaborn build upon Matplotlib \citep[the default Python plotting library;][]{Hunter2007} and were used to generate many of the figures. Iris was also used for rotating the global coordinate system and meridional wind (via the PROJ.4 Cartographic Projections Library), and the pyqt\_fit, eofs and windspharm libraries were used for kernel density estimation, EOF analysis and for calculating the streamfunction respectively.

An accompanying Figshare repository has been created to document the computational methodology in more detail \citep{Irving2016a}. It contains the specifics of the software packages discussed above (i.e. version numbers, release dates, web addresses) as well as a supplementary log file for each figure in the paper. Those log files outline the computational steps performed from initial download of the ERA-Interim data through to the final generation of the plot, and a version controlled repository of the relevant code can be found at \url{https://github.com/DamienIrving/climate-analysis}. The rationale behind this approach to computational reproducibility is explained by \citet{Irving2016}.