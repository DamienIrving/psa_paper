\subsection{The PSA pattern}

The relative frequency of the positive (defined as phase 8-18$^{\circ}$) and negative (39-49$^{\circ}$) phase of the PSA pattern has varied over the period 1979-2014. During autumn and winter in particular, the middle years (~1991-2002) were characterized by a predominance of positive phase events, while negative phase events have been more common in recent years (FIGURE 7). This variability is reflected in the linear trends observed over that time, with negative phase events showing a statistically significant increasing trend (at the $p < 0.1$ level) both annually and during autumn (FIGURE 8 - mention the statistically significant trends in the caption). Consistent with previous studies, the PSA-pattern showed a preference for winter and spring (FIGURE 8).





In the analysis of the PSA pattern that follows we extend the filter to include values $\pm$5$^{\circ}$ of the local maxima, which helps to account for events that propagate and in the process likely captures the (often degenerate) PSA-2 mode. 

Results differ for different epochs but not for stationary versus propagating waves (the backwards have a flat distribution but forwards and stationary have the same shape as the overall.

Phase subsets: seasonal trends, ENSO/SAM relationship, tas/pr/sic composites