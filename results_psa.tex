\subsection{The PSA pattern}

The relative frequency of the positive (defined as phase 8-18$^{\circ}$) and negative (39-49$^{\circ}$) phase of the PSA pattern has varied over the period 1979-2014. During autumn and winter in particular, the middle years (\~1991-2002) were characterized by a predominance of positive phase events, while negative phase events have been more common in recent years (FIGURE 7). This variability is reflected in the linear trends observed over that time, with negative phase events showing a statistically significant increasing trend (at the $p < 0.1$ level) both annually and during autumn (FIGURE 8 - mention the statistically significant trends in the caption). Consistent with previous studies, the PSA-pattern showed a preference for winter and spring (FIGURE 8).

In order to assess the influence of the PSA pattern on regional climate variability, the composite mean surface air temperature anomaly, precipitation anomaly and sea ice concentration anomaly was calculated for both the positive and negative phase (FIGURE 9). On the western flank of the central streamfunction anomaly associated with positive phase events, anomalously warm conditions were evident over the Ross Sea, Amundsen Sea and interior of West Antactica, particularly during autumn and winter. The northerly flow responsible for those warm conditions also induced large precipitation increases along the West Antarctic coastline and reduced sea ice in the Amundsen Sea. On the eastern flank, anomalously cool conditions were evident over the Antarctic Peninsula, Patagonia and the Weddell Sea during all seasons (winter and spring especially), with the latter also experiencing large increases in sea ice. Anomalously dry conditions were also seen over the Antarcitc Peninsula in association with the weaker westerly flow. 


ENSO/SAM relationship

Results differ for different epochs but not for stationary versus propagating waves (the backwards have a flat distribution but forwards and stationary have the same shape as the overall.