\subsection{The PSA pattern}

In defining the PSA pattern according to the peaks of the PSA-like phase distribution, it was necessary to account for seasonal variations in the location of those peaks (Figure \ref{fig:phase_distribution}). A spread of 15$^{\circ}$E was considered sufficient to capture those variations, and hence the 1979-2014 annual Guassian kernel density estimate used to isolate the 15$^{\circ}$ interval about the two local maxima containing the highest local values (i.e. to account of the fact that the density estimates were not symmetrical about the local maxima). This yielded two intervals corresponding to the positive (4.5-19.5$^{\circ}$E) and negative (37.5-52.5$^{\circ}$E) phase of the PSA pattern.

During autumn and winter in particular, the middle years (\~1991-2002) were characterized by a predominance of positive phase events, while negative phase events have been more common in recent years (Figure \ref{fig:phase_distribution}). This variability is reflected in the linear trends observed over that time, with negative phase events showing a statistically significant increasing trend (at the $p < 0.05$ level) on an annual basis (Figure \ref{fig:psa-neg-seasonality}). Consistent with previous studies, the PSA pattern showed a preference for winter and spring (Figure \ref{fig:psa-pos-seasonality} and \ref{fig:psa-neg-seasonality}).

In order to assess the influence of the PSA pattern on regional climate variability, the composite mean surface air temperature anomaly, precipitation anomaly and sea ice concentration anomaly was calculated for both the positive and negative phase (Figure \ref{fig:surface_composites}). On the western flank of the central streamfunction anomaly associated with positive phase events, anomalously warm conditions were evident over the Ross Sea, Amundsen Sea and interior of West Antactica, particularly during autumn and winter. The northerly flow responsible for those warm conditions also induced large precipitation increases along the West Antarctic coastline and reduced sea ice in the Amundsen Sea. On the eastern flank, anomalously cool conditions were evident over the Antarctic Peninsula, Patagonia and the Weddell Sea during all seasons (winter and spring especially), with the latter also experiencing large increases in sea ice. Anomalously dry conditions were also seen over the Antarcitc Peninsula in association with the weaker westerly flow. 

Other things to talk about:
\begin{itemize}
\item ENSO/SAM relationship (ad a neutral area to the plot - want to know what percentage is not associated with strong ENSO). The SAM/ENSO coupling discussed by \citet{Fogt2006} is evident, but not super pronounced and doesn't explain the trends.
\item Results differ for different epochs but not for stationary versus propagating waves (the backwards have a flat distribution but forwards and stationary have the same shape as the overall).
\item The pronounced positive streamfunction anomaly over the Indian Ocean in the PSA pattern composites gives the impression of a hemispheric zonal wavenumber three pattern, however the phase of that pattern and the unremarkable negative anomalies either side Indian Ocean anomaly are inconsistent with the characteristics of the dominant SH zonal wavenumber three mode \citep[e.g.][]{Raphael2004,IrvingSimmonds2015}.
\end{itemize}


