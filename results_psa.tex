\subsection{The PSA pattern}\label{s:psa_results}

In defining the PSA pattern according to the peaks of the PSA-like phase distribution, it was necessary to account for seasonal variations in the location of those peaks (Figure \ref{fig:phase_distribution}). A spread of 15$^{\circ}$ was considered sufficient to capture these variations and hence the 15$^{\circ}$ interval about each local maxima containing the highest mean values (taken from the annual kernel density estimate) was determined. This approach was used to account of the fact that the phase histograms were not symmetrical about the local maxima and it yielded two intervals corresponding to the positive (4.5-19.5$^{\circ}$E) and negative (37.5-52.5$^{\circ}$E) phase of the PSA pattern. Both intervals represented approximately 15\% of all data times (14.8\% for the positive phase versus 15.8\% for the negative), which suggests that the two phases have a similar frequency of occurrence. With this definition in place, it was possible to investigate variability and trends in the PSA pattern as well as its influence on surface temperature, precipitation and sea ice. 

\subsubsection{Trends and variability}

During autumn and winter in particular, the middle years of the study period (1991-2002) were characterized by a predominance of positive PSA pattern activity, while negative phase activity was more common in recent years (Figure \ref{fig:phase_distribution}). This variability is reflected in the linear trends observed over 1979-2014, with negative phase activity showing a statistically significant increasing trend (at the $p < 0.05$ level) on an annual basis and smaller non-significant increasing trends for summer, autumn and winter (Figure \ref{fig:psa-neg_seasonality}). Positive phase activity showed a non-significant decreasing trend on an annual basis and also during autumn and winter, with an increasing trend observed for summer (Figure \ref{fig:psa-pos_seasonality}). Consistent with previous studies, both phases of the PSA pattern were most active during winter and spring (Figure \ref{fig:psa-neg_seasonality} and \ref{fig:psa-pos_seasonality}). 

In attempting to explain annual and decadal variability in PSA pattern activity, previous authors have suggested that coupling between the SAM and ENSO is important \citep[e.g.][]{Fogt2006}. While some degree of coupling is evident in Figure \ref{fig:sam_v_enso} (i.e. the positive phase of the PSA pattern was most common when positive ENSO events and negative SAM events coincided), it is clear that the SAM has a much stronger association with PSA pattern activity than ENSO. Given that summer is the only season during which there is a statistically significant trend in the SAM during recent decades, it does not appear that variability in either the SAM or ENSO are the major drivers of the trends recorded in PSA pattern activity.


\subsubsection{Influence on surface variables} 

In order to assess the influence of the PSA pattern on regional climate variability, the composite mean surface air temperature anomaly, precipitation anomaly and sea ice concentration anomaly was calculated for both the positive and negative phase (Figure \ref{fig:surface_composites}). On the western flank of the central composite-mean streamfunction anomaly associated with positive phase activity, anomalously warm conditions were evident over the Ross Sea, Amundsen Sea and interior of West Antactica, particularly during autumn and winter. The northerly flow responsible for those warm conditions also induced large precipitation increases along the West Antarctic coastline and reduced sea ice in the Amundsen Sea. On the eastern flank, anomalously cool conditions were evident over the Antarctic Peninsula, Patagonia and the Weddell Sea during all seasons (winter and spring especially), with the latter also experiencing large increases in sea ice. Anomalously dry conditions were also seen over the Antarctic Peninsula in association with the weaker westerly flow. 

The anomalies associated with the negative phase of the PSA pattern were essentially the reverse of the positive phase (Figure \ref{fig:surface_composites}). It is also noteworthy that while the hemispheric composite-mean streamfunction anomaly associated with the PSA pattern gives the impression of a hemispheric zonal wavenumber three pattern, the phase of that pattern and the unremarkable anomalies either side of the Indian Ocean anomaly are inconsistent with the characteristics of the dominant SH zonal wavenumber three mode \citep[e.g.][]{Raphael2004,IrvingSimmonds2015}.


