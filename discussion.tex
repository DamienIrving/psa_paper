\section{Discussion}

A novel methodology has been presented for objectively identifying the PSA pattern. By rotating the global coordinate system such that the equator (a great circle path) traces the approximate path of the PSA pattern, the method was able to utilize Fourier analysis to quantify the phase and amplitude of wave-like variability in the PSA sector. The climatology produced from the application of this method revealed that the PSA pattern often persists for months at a time. There was an almost equal split between events that propagate (to the east) and those that remain relatively stationary, and the pattern is most active during winter and spring. The pattern has a strong influence on temperature and precipitation variability over West Antartica and the Antarctic Peninsula, and on sea ice variability in the adjacent Amundsen, Bellingshausen and Weddell Seas. 

In reconciling the results of the Fourier analysis with existing EOF-based definitions of the PSA pattern, a strong resemblance was found between the existing PSA-1 mode and the spatial pattern identified here corresponding to the bimodal phase peaks of wavenumber 5-6 dominant variability in the PSA sector. The lack of a higher-order, multi-modal phase distribution may explain the degenerate nature of the existing PSA-2 mode (and the difficulty that researchers have had in identifying a tropical driver of that mode). It would appear that together, the degenerate EOF-2 / EOF-3 pair (e.g. Figure \ref{fig:eof}) may simply represent the remaining wavenumber 5-6 variability in the PSA sector, which likely comprises of contributions from the hemispheric zonal wavenumber three pattern as well as isolated Amundsen Sea Low and Antarctic Dipole variability.    
By using the bimodal phase peaks as a means to define PSA pattern activity, it was possible to identify a trend towards the negative phase of the pattern over the period 1979-2014 on an annual basis and also during autumn and winter. This autumn trend (and the high latitude temperature and sea ice anomalies associated with the negative phase of the PSA pattern) is consistent with the work of \citet{Ding2013}, who found that autumn warming over the Antarctic Peninsula and associated sea ice declines over the Bellingshausen Sea are associated with an atmospheric circulation resembling the negative phase of the PSA pattern. While this explanation makes sense on the eastern flank of the central circulation anomaly associated with that pattern, the negative phase of the PSA pattern is also associated with strong cooling over West Antarctica. Autumn temperature declines have not been observed in that region, which means that if our results are to be believed, then the PSA-related cooling must have been offset by other factors. 

In contrast to the autumn warming over the Antarctic Peninsula, winter warming over West Antarctica has been associated with an atmospheric circulation resembling the positive phase of the PSA pattern \citep{Ding2011}. Our climatology revealed a non-significant trend towards the negative phase of the PSA pattern during winter, which raises the question: how is it that winter temperature trends over West Antarctica are associated with an atmospheric circulation resembling the positive phase of the PSA pattern, but a climatology of PSA pattern activity does not reveal trends consistent with that finding? The answer to this question may have been uncovered by \citet{Li2015a}. They analyzed Rossby wave trains associated with observed SST trends in the tropical Atlantic, tropical Indian, west Pacific and east Pacific regions and found that all four have a center of action over the Amundsen Sea. While none of these individual wave trains resembled the PSA pattern, a linear combination of the four of them did (with the tropical Atlantic and west Pacific identified as most influential). In other words, the integrated influence of tropical SST trends on the atmospheric circulation resembles the positive phase of the PSA pattern, but the waves underpinning that teleconnection do not. This result is consistent with earlier studies that identified the tropical Atlantic as a driver of recent trends in West Antarctica \citep{Li2014,Simpkins2014} and goes to the heart of the argument made at the beginning of this paper. For a proposed teleconnection to be robust, it must be evident when looking through the lens of both the variable and mechanism of interest.

While this explanation appears to reconcile the discrepancy between our climatology and winter warming over West Antarcitca, the associated circulation anomaly would bring cooler conditions and wind-driven increases in sea ice along the western Antarctic Peninsula, contrary to the observed warming and decreases in sea ice extent there \citep{Clem2015}. One possible explanation is that the negative autumn sea ice anomalies persist into winter \citet{Ding2013}, however it is clear that there is still work to be done to fully understand recent temperature and sea ice changes in the region.

One topic not addressed by our climatology is variability in the east/west location of the PSA pattern. In response to the emergence of central Pacific ENSO events in recent years, some authors have suggested that the PSA pattern moves east/west depending on the precise location of the associated tropical SST anomalies \citep[e.g.][]{Sun2013,Wilson2014,Ciasto2015}. Others suggest that the pattern is relatively stationary \citep[e.g.][]{Liu2007,Ding2012}, however either way the broad region (10$^{\circ}$N to 10$^{\circ}$S in the rotated coordinate system) used by our identification algorithm renders it insensitive to subtle east/west movements. Given that the PSA pattern did not show a strong association with the Ni\~{n}o 3.4 index (an index that is sensitive to both central and eastern Pacific ENSO events), it would be fair to say that even if the location of tropical SSTs does cause the pattern to move slightly, this would represent only a small fraction of all PSA pattern events. 

This weak association with ENSO challenges our fundamental understanding of the PSA pattern. The most commonly held view to date is that the pattern is primarily a response to ENSO forcing \citep[e.g.][]{Mo2001} that is moderated by the state of the `atmospheric bridge' (BRIDGE REFS). In particular, the pattern is thought to be most active when ENSO and the SAM are in phase \citep{Fogt2006}. A more comprehensive analysis of the relationship between the pattern and tropical convection would be required to confirm this (e.g. lagged correlations with SSTs and other indicators of tropical convection like the outgoing longwave radiation), but our results suggest that the PSA pattern might be better conceptualized as a preferred regional atmospheric response to various internal and external forcings (i.e. with ENSO being just one of many players). Rather than the SAM playing a facilitating/bridging role, the strong association identified here is more consistent with the work of \citep{Ding2012} who essentially suggest that the PSA pattern \textit{is} the SAM in the eastern hemisphere. 

In addition to a more detailed analysis of the relationship between the PSA pattern and both tropical convection and the SAM, our new methodology could also be adapted for studies of other quasi-stationary waveforms. The most obvious candidate is the Pacific-North American (PNA) pattern \citep{Wallace1981}, which plays an important role in winter climate variability over the North Pacific and North America \citep[e.g.][]{Notaro2006}. Like its namesake, the PNA pattern follows an approximate great circle path, has traditionally been analyzed via EOF analysis and has been implicated in recent mid-to-high latitude trends \citep[e.g.][]{Ding2014,Liu2015}. Other non-zonal waveforms that do not follow an approximate great circle path would be more challenging, however methods have been developed for applying Fourier analysis to synoptic-scale, non-zonal waveforms \citep{Zimin2006} and may represent a starting point for future research. 
