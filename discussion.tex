\section{Discussion}

A novel method for identifying the PSA pattern has been developed and applied to the problem of characterizing its climatological characteristics. By rotating the global coordinate system such that the equator (a great circle path) traces the approximate path of the PSA pattern, Fourier analysis could be used   


Most of the climate variability impacts are documented really for the first time.

The positive trend in the negative phase of the PSA pattern during autumn is consistent with the work of \citet{Ding2013}, who found that warming over the Antarctic Peninsula during that season is associated with a PSA-like circulation (their story about the reduced sea ice is consistent too). That pattern, however, is also associated with strong cooling over central Western Antarctic, which hasn't been observed (it must be offset by something else). In fact, there was a weak positive trend in negative phase events during winter as well, which is in direct contrast to the strong warming seen in winter throughout West Antarctica. How is it then that recent high latitude trends are associated with an atmospheric circulation that resembles the PSA pattern, but a climatology of PSA pattern activity does not reveal trends consistent with that finding? For winter at least, the answer to that question may have been uncovered by \citet{Li2015a}. They analyze the Rossby wave trains associated with observed SST trends in the tropical Atlantic, tropical Indian, west Pacific and east Pacific regions and find that all four have a center of action over the Amundsen Sea. While none of these individual wave trains resemble the PSA pattern, a linear combination of the four of them does (with the tropical Atlantic and west Pacific identified as most influential). In other words, the integrated influence of tropical SST trends on the atmospheric circulation resembles the PSA pattern, but the waves underpinning that teleconnection do not. This result is consistent with earlier studies that identified the possible influence of the tropical Atlantic on West Antartic trends \citep{Li2014,Simpkins2014} and goes to the heart of the argument made at the beginning of this paper: for a proposed teleconnection to be robust, it must go both ways. 

From Clem and Fogt (2015): According to Ding2011, increasing SSTs in the central tropical Pacific generate a Rossby wave train similar to the PSA, which produces positive geopotential height anomalies in the Amundsen Sea in winter, driving increased warm air advection onto West Antarctic on its western flank. This scenario would be associated with cooling and wind-driven increases in sea ice along the wester Antarctic Peninsula in winter, contrary to the observed warming and decreases in sea ice extent there. Therefore, the high southern latitude regional circulation changes associated with tropical Pacific SSts are not fully understood...

Central Pacific events are better at inducing the PSA pattern and make it move. I can look at the central Pacific thing maybe, but not whether they move. My technique isn't sensitive enough and clearly ENSO forced patterns represent only a small fraction of the total sample.

From Clem2013: Fogt et al. [2011] found that the strength of the ENSO teleconnection to the South Pacific is governed by the coupling of ENSO with the SAM. When an El Nino (La Nina) event occurs with a negative (positive) SAM event, the two climate patterns are said to be ‘in phase’, and the ENSO teleconnection to the South Pacific is stronger than average. Fogt et al. [2011] also noted that when ENSO and SAM are ‘out of phase’ [when an El Nino (La Nina) event occurs with a positive (negative) SAM event], the teleconnection is significantly weakened, displaced, or altogether absent. Using this ENSO-SAM relationship, Fogt and Bromwich [2006] determined that the overall weak ENSO teleconnection to the South Pacific in SON during the 1980s was due to a negative correlation between the Southern Oscillation Index (SOI) and the SAM index, indicating an out-of-phase relationship between these two modes. In the 1990s, the correlation became significantly positive, and as a result, the teleconnection was amplified.


Further work: Lagged composites for tropical drivers, PNA, more detailed account of PSA like Kosaka, SAM / PSA relationship (better Ding2012)