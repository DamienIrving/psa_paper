\section{Discussion}

A novel methodology has been presented for objectively identifying the PSA pattern. By rotating the global coordinate system such that the equator (a great circle path) traces the approximate path of the PSA pattern, the method was able to utilize Fourier analysis to as opposed to a traditional EOF-based approach. The bimodal phase distribution goes some way to explaining the degenerate nature of the PSA-2 mode - it and EOF3 simply capture rest of the PSA-like variability happening in the PSA-sector.


The climatology produced from the application of this method revealed that the PSA pattern often persists for months at a time. There was an almost equal split between events that propagate (to the east) and those that remain relatively stationary, and the pattern is most active during winter and spring. The pattern has a strong influence on temperature and precipitation variability over West Antartica and the Antarctic Peninsula, and on sea ice variability in the adjacent Amundsen, Bellingshausen and Weddell Seas. 

A trend towards the negative phase of the PSA pattern was observed over the period 1979-2014 on an annual basis and also during autumn. This autumn trend (and the high latitude temperature and sea ice anomalies associated with the negative phase of the PSA pattern) is consistent with the work of \citet{Ding2013}, who found that autumn warming over the Antarctic Peninsula and associated sea ice declines over the Bellingshausen Sea are associated with an atmospheric circulation resembling the negative phase of the PSA pattern. The negative phase of the PSA pattern is also associated with strong cooling over West Antarctica, so it is noteworthy that autumn temperature declines have not been observed in that region (i.e. if our results are to be believed, then the PSA-related cooling must have been offset by other factors). 

In contrast to the autumn warming over the Antarctic Peninsula, winter warming over West Antarctica has been associated with an atmospheric circulation resembling the positive phase of the PSA pattern \citep{Ding2011}. Our climatology revealed a non-significant trend towards the negative phase of the PSA pattern during winter, which raises the question: how is it that winter temperature trends over West Antarctica are associated with an atmospheric circulation resembling the positive phase of the PSA pattern, but a climatology of PSA pattern activity does not reveal trends consistent with that finding? The answer to this question may have been uncovered by \citet{Li2015a}. They analyzed Rossby wave trains associated with observed SST trends in the tropical Atlantic, tropical Indian, west Pacific and east Pacific regions and found that all four have a center of action over the Amundsen Sea. While none of these individual wave trains resembled the PSA pattern, a linear combination of the four of them did (with the tropical Atlantic and west Pacific identified as most influential). In other words, the integrated influence of tropical SST trends on the atmospheric circulation resembles the positive phase of the PSA pattern, but the waves underpinning that teleconnection do not. This result is consistent with earlier studies that identified the tropical Atlantic as a driver of recent trends in West Antarctica \citep{Li2014,Simpkins2014} and goes to the heart of the argument made at the beginning of this paper. For a proposed teleconnection to be robust, it must be evident when looking through the lens of both the variable and mechanism of interest.

While this explanation appears to reconcile the discrepancy between our climatology and winter warming over West Antarcitca, the associated circulation anomaly would bring cooler conditions and wind-driven increases in sea ice along the western Antarctic Peninsula, contrary to the observed warming and decreases in sea ice extent there \citep{Clem2015}. One possible explanation is that the negative autumn sea ice anomalies persist into winter \citet{Ding2013}, however it is clear that there is still work to be done to fully understand recent temperature and sea ice changes in the region.

One reason for the observed trends in the PSA pattern 
The reason for those trends might be SAM/ENSO coupling: From Clem2013: Fogt et al. [2011] found that the strength of the ENSO teleconnection to the South Pacific is governed by the coupling of ENSO with the SAM. When an El Nino (La Nina) event occurs with a negative (positive) SAM event, the two climate patterns are said to be ‘in phase’, and the ENSO teleconnection to the South Pacific is stronger than average. Fogt et al. [2011] also noted that when ENSO and SAM are ‘out of phase’ [when an El Nino (La Nina) event occurs with a positive (negative) SAM event], the teleconnection is significantly weakened, displaced, or altogether absent. Using this ENSO-SAM relationship, Fogt and Bromwich [2006] determined that the overall weak ENSO teleconnection to the South Pacific in SON during the 1980s was due to a negative correlation between the Southern Oscillation Index (SOI) and the SAM index, indicating an out-of-phase relationship between these two modes. In the 1990s, the correlation became significantly positive, and as a result, the teleconnection was amplified.

One topic that was not addressed by our climatology was the relationship between the PSA pattern and the location of SST anomalies in the tropical Pacific.  

Related to ENSO is the work that says central Pacific events are better at inducing the PSA pattern and make it move. I can look at the central Pacific thing maybe, but not whether they move. My technique isn't sensitive enough and clearly ENSO forced patterns represent only a small fraction of the total sample.




Further work: Lagged composites for tropical drivers, PNA, more detailed account of PSA like Kosaka, SAM / PSA relationship (better Ding2012)