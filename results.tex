\section{Results}

In applying the identification algorithm to the ERA-Interim dataset, the decision was made to focus on monthly timescale data at 500 hPa. The latter represents a mid-to-upper tropospheric level that is below the tropopause in all seasons and at all latitudes of interest. Given the equivalent barotropic nature of the PSA pattern (i.e. the wave amplitude increases with height but phase lines tend to be vertical) the results do not differ substantially for other levels of the troposphere. The monthly timescale was selected to be consistent with most previous studies of the PSA pattern,  In order to be consistent with previous studies, 


By applying the Fourier transform used in the identification process to 500 hPa rotated meridional wind anomaly data, it was possible to investigate the implications of the timescale selection (FIGURE 4). That analysis revealed wavenumber 7 as the most dominant frequency for daily timescale data in the PSA-sector, with wavenumber 6 dominating the frequency spectrum for a 10- through to 90-day running window. 