\section{Results}

\subsection{General PSA-like variability}

Before attempting to isolate the PSA pattern via the phase information obtained from the identification algorithm, it is worth considering the characteristics of all PSA-like variability identified in the PSA-sector. In total, 55\% (7163 of 13120) of data times were identified by the algorithm (i.e. wavenumber 5 and 6 were among the top three ranked frequencies), which is consistent with the fact that wavenumber 6 dominates the Fourier spectrum at the monthly timescale (FIGURE 4). Grouping consecutive identifications into events revealed a median event duration of X data times, with a distribution depicted in FIGURE 5. Events of shorter duration tended to occur either side of longer events, as the coordinated wave pattern associated with the longer event began to form/decay.

The lifecycle of events lasting longer than 10 data times is depicted in FIGURE 6. The amplitude of most events tended to peak

Duration, propagation, phase distribution (with spatial plots for the local minima and maxima).

\subsection{The PSA pattern}

Phase subsets: seasonal trends, ENSO/SAM relationship, tas/pr/sic composites







