\section{Results}

\subsection{General PSA-like variability}

Before attempting to isolate the PSA pattern via the phase information obtained from the identification algorithm, it is worth considering the characteristics of all PSA-like variability identified in the PSA-sector. In total, 55\% (7163 of 13120) of data times were identified as displaying PSA-like variability (i.e. wavenumber 5 and 6 were among the top three ranked frequencies), which is consistent with the fact that wavenumber 6 dominates the Fourier spectrum at the monthly timescale (FIGURE 4). Grouping consecutive identifications into discrete events revealed a median event duration of X data times, with a distribution depicted in FIGURE 5. While interpretation of these duration data is somewhat confused by the 30-day running mean applied to the original data (e.g. an event that lasted 10 data times could be said to span anywhere between 10 and 40 days) and also the occurrence of short events immediately before or after a long event (i.e. they could conceivably be considered as a single event), it is clear that PSA-like variability can often persist for up to a few months at a time.     

Buidling on this baseline duration data, the life cycle of events lasting longer than 10 data times was investigated in more detail (FIGURE 6). The amplitude of these events tended to peak mid-event, with some longer-lasting events peaking more than once during their lifetime (perhaps indicating that some events simply merge into the next). The phase information associated with these events reveals that there was an approximately even split between those that we almost stationary throughout their lifetime (defined as a linear phase trend of magnitude less than 0.2 degrees longitude per data time) and those that propagated towards the east on the rotated grid (linear trend greater than 0.2 degrees longitude per data time). This corresponds to the direction from east of New Zealand towards the Amundsen and then Weddell Sea on a regular coordinate system. 

By looking at the phase information 


Duration, propagation, phase distribution (with spatial plots for the local minima and maxima).

\subsection{The PSA pattern}

Phase subsets: seasonal trends, ENSO/SAM relationship, tas/pr/sic composites







