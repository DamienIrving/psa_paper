\section{Results}

\subsection{General PSA-like variability}

Before attempting to isolate the PSA pattern via the phase information obtained from the identification algorithm, it is worth considering the characteristics of all PSA-like variability identified in the PSA-sector. In total, 55\% (7163 of 13120) of data times were identified as displaying PSA-like variability (i.e. wavenumber 5 and 6 were among the top three ranked frequencies), which is consistent with the fact that wavenumber 6 dominates the Fourier spectrum at the monthly timescale (FIGURE 4). Grouping consecutive identifications into discrete events revealed a median event duration of X data times, with a distribution depicted in FIGURE 5. While interpretation of these duration data is somewhat confused by the 30-day running mean applied to the original data (e.g. an event that lasted 10 data times could be said to span anywhere between 10 and 40 days) and the occurrence of short events immediately before or after a long event (i.e. they could conceivably be considered as a single event), it is clear that PSA-like variability can often persist for up to a few months at a time.     

The lifecycle of events lasting longer than 10 data times is depicted in FIGURE 6. The amplitude of most events tended to peak mid-event, while some longer-lasting events peaked more than once throughout their lifetime. The phase information reveals that the majority of events propagated towards the east on the rotated grid COMMENT ON MEAN SPEED, which corresponds to the direction from east of New Zealand towards the Amundsen and then Weddell Sea on a regular coordinate system. 

By looking at the phase information 


Duration, propagation, phase distribution (with spatial plots for the local minima and maxima).

\subsection{The PSA pattern}

Phase subsets: seasonal trends, ENSO/SAM relationship, tas/pr/sic composites







