\section{Results}

\subsection{General PSA-like variability}

Before attempting to isolate the PSA pattern via the phase information obtained from the identification algorithm, it is worth considering the characteristics of all PSA-like variability identified in the PSA-sector. In total, 55\% (7163 of 13120) of data times were identified as displaying PSA-like variability (i.e. wavenumber 5 and 6 were among the top three ranked frequencies), which is consistent with the fact that wavenumber 6 dominates the Fourier spectrum at the monthly timescale (FIGURE 4). Grouping consecutive identifications into discrete events revealed a median event duration of X data times, with a distribution depicted in FIGURE 5. While interpretation of these duration data is somewhat confused by the 30-day running mean applied to the original data (e.g. an event that lasted 10 data times could be said to span anywhere between 10 and 40 days) and also the occurrence of short events immediately before or after a long event (i.e. they could conceivably be considered as a single event), it is clear that PSA-like variability often persists for up to a few months at a time.     

Buidling on this baseline duration data, the life cycle of events lasting longer than 10 data times was investigated in more detail (FIGURE 6). The amplitude of these events tended to peak mid-event, with some longer-lasting events peaking more than once during their lifetime (perhaps indicating that some events simply merge into the next). The phase information associated with these events reveals that there was an approximately even split between those that were almost stationary throughout their lifetime (defined as a linear phase trend of magnitude less than 0.2$^{\circ}$E per data time) and those that propagated towards the east on the rotated grid (linear trend greater than 0.2$^{\circ}$E per data time) TRY DIFFERENT GRADIENT DEFINITIONS??. This propagation towards the east in the rotated coordinate system corresponds to the Amundsen Sea to Weddell Sea direction along the PSA-sector on a regular coordinate system. 

Important insights can also be gained by looking at the phase distribution for all individual PSA-like data times (FIGURE 7). On an annual basis the distribution is clearly bimodal, with the two maxima located at approximately 13$^{\circ}$E and 44$^{\circ}$E. Since the phase was defined as the location of the first local maxima of the wavenumber 6 component of the Fourier transform, this 30$^{\circ}$ phase separation indicates that the spatial patterns associated with these peaks are essentially exactly out of phase (FIGURE 8). These patterns also closely resemble the positive and negative polarity of the PSA-1 pattern identified by previous authors, hence filtering the PSA-like data times according to the location of the two local maxima represents a useful approach to isolating the PSA pattern from the larger population of PSA-like variability. Extending the filter to include values  $\pm$5$^{\circ}$ will go some way to accounting for events that propagate and in the process capture the (often degenerate) PSA-2 mode. 

Results differ for different epochs but not for stationary versus propagating waves (the backwards have a flat distribution but forwards and stationary have the same shape as the overall.

Also interesting to look at the minima. Might represent the ASL? Certainly indicative of a non-complete pattern.

\subsection{The PSA pattern}

Phase subsets: seasonal trends, ENSO/SAM relationship, tas/pr/sic composites







