The Pacific-South American (PSA) pattern is an important mode of climate variability in the mid-to-high southern latitudes. It is widely recognized as the primary mechanism by which the El Ni\~{n}o Southern Oscillation (ENSO) influences the south-east Pacific and south-west Atlantic, and in recent years has also been suggested as a mechanism by which longer-term tropical sea surface temperature trends can influence the Antarctic climate. Despite this recognition, relatively little is known about the climatological characteristics of the pattern. This issue is addressed here by the development and application of a novel methodology for objectively identifying the PSA pattern from ERA-Interim reanalysis data. By rotating the global coordinate system such that the equator (a great circle) traces the approximate path of the pattern, the identification algorithm utilizes Fourier analysis as opposed to a traditional Empirical Orthogonal Function approach. The resulting climatology reveals that the PSA pattern has a strong influence on temperature and precipitation variability over West Antarctica and the Antarctic Peninsula, and on sea ice variability in the adjacent Amundsen, Bellingshausen and Weddell Seas. An identified trend towards the negative polarity of the PSA pattern during autumn is consistent with warming observed over the Antarctic Peninsula during that season, however a similar (but not significant) trend during winter is inconsistent with the warming observed over West Antarctica. Only a weak relationship is identified between the PSA pattern and ENSO, which suggests that the pattern is a preferred atmospheric response to various external (and internal) forcings.