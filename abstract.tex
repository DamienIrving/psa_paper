The Pacific-South American (PSA) pattern is an important mode of climate variability in the mid-to-high southern latitudes. It is widely recognized as the primary atmospheric teleconnection by which the El Ni\~{n}o Southern Oscillation (ENSO) influences the south-east Pacific and south-west Atlantic and has been implicated in recent climatic trends over West Antartica and the Antarctic Peninsula. Despite this recognition, relatively little is known about the climatological characteristics of the pattern. We seek to address this issue via the development and application of a novel methodology for objectively identifying the pattern in ERA-Interim reanalysis data. By rotating the global coordinate system such that the equator (a great circle) traces the approximate path of the PSA pattern, our identification algorithm is able to utilize Fourier analysis as opposed to traditional Empirical Orthogonal Function approaches. We find that the PSA pattern often persists for months at a time, with an almost equal split between events that propagate (to the east) and those that remain stationary. The pattern has a strong influence on temperature and precipitation variability over West Antarctica and the Antarctic Peninsula, and on sea ice variability in the adjacent Amundsen, Bellingshausen and Weddell Seas. Trends towards the negative polarity of the pattern during autumn are consistent with  