The Pacific-South American (PSA) pattern is an important mode of climate variability in the mid-to-high southern latitudes. It is widely recognized as the primary mechanism by which the El Ni\~{n}o Southern Oscillation (ENSO) influences the south-east Pacific and south-west Atlantic, and in recent years has also been identified as a possible linking teleconnection between trends in those regions and tropical sea surface temperatures. Despite this recognition, relatively little is known about the climatological characteristics of the pattern. This study addresses this issue via the development and application of a novel methodology for objectively identifying the pattern from ERA-Interim reanalysis data. By rotating the global coordinate system such that the equator (a great circle) traces the approximate path of the PSA pattern, the identification algorithm utilizes Fourier analysis as opposed to traditional Empirical Orthogonal Function approaches. The resulting climatology reveals that the PSA pattern has a strong influence on temperature and precipitation variability over West Antarctica and the Antarctic Peninsula and on sea ice variability in the adjacent Amundsen, Bellingshausen and Weddell Seas. The identified trend towards the negative polarity of the PSA pattern during autumn is consistent with warming observed over the Antarctic Peninsula during that season, however a similar (but not significant) trend during winter is inconsistent with warming observed over West Antarctica. Only a weak relationship is identified between the PSA pattern and ENSO, consistent internal atmospheric fluctuations play a more important role in 