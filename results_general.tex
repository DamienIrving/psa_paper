\section{Results}

\subsection{General PSA-like variability}

Before attempting to isolate the PSA pattern using the phase information obtained from the identification algorithm, it is worth considering the characteristics of all PSA-like variability. In total, 55\% (7163 of 13120) of data times were identified as displaying PSA-like variability (i.e. wavenumber 5 and 6 were among the top three ranked frequencies), which is consistent with the fact that wavenumber 6 dominates the Fourier spectrum at the monthly timescale (Figure \ref{fig:periodogram}). Grouping consecutive identifications into discrete events revealed a mean event duration of 19.7 data times, with a distribution depicted in Figure \ref{fig:lifecycle}a. While interpretation of these duration data is complicated by the 30-day running mean applied to the original data (e.g. an event that lasted 10 data times could be said to span anywhere between 10 and 40 days) and the occurrence of short events immediately before or after a long event (i.e. they could conceivably be considered as a single event), it is clear that PSA-like variability often persists for up to a few months at a time.     

Buidling on this baseline duration data, the life cycle of events lasting longer than 10 data times was investigated in more detail. As depicted in Figure \ref{fig:lifecycle}b, the amplitude of these events tended to peak mid-event with some longer-lasting events peaking more than once during their lifetime (perhaps suggesting that some events simply merge into the next). The mean ($\pm$ standard deviation) linear phase trend across all events lasting longer than 10 data times was $0.12 \pm 0.38^{\circ}$E per data time, which indicates that while there was a tendency for events to propagate to the east, a substantial proportion moved very little during their lifetime. Important insights were also obtained by considering the phase distribution across all individual PSA-like data times (Figure \ref{fig:phase_distribution}). On an annual basis the distribution is clearly bimodal, with the two maxima of the kernel density estimate located at 12.75$^{\circ}$E and 45.0$^{\circ}$E. Since the phase was defined as the location of the first local maxima of the wavenumber 6 component of the Fourier transform, this approximate 30$^{\circ}$ phase separation indicates a pair of spatial patterns that are exactly out of phase (Figure \ref{fig:sf_composites}). Taken together these patterns clearly represent the single most dominant mode of variability in the PSA sector, and also closely resemble the PSA-1 mode identified by previous authors. Unlike the spatial patterns associated with the local minima (Figure \ref{fig:sf_composites}), the spatial streamfunction anomalies associated with these maxima are of similar magnitude, which suggests that the underlying data times tended to display a coordinated wave pattern as opposed to an individual anomaly of wavenumber 5-6 scale. On the basis of this finding, it appears that filtering the PSA-like data times according to the location of the two local maxima represents a simple and valid technique for isolating the PSA pattern from the larger population of PSA-like variability. 


