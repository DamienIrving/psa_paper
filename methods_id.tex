\section{Methodology}

Many existing climatologies of Rossby wave activity focus on zonally propagating waves, whereby the wave identification method simply applies a Fourier transform along adjacent lines of constant latitude \citep[e.g.][]{Glatt2014,IrvingSimmonds2015}. More sophisticated methods have been developed for tracking non-zonal synoptic-scale Rossby wave packets \citep[e.g.][]{Zimin2006,Souders2014}, however such sophistication was not needed in devising a method for objectively identifying the PSA pattern. Instead, it was possible to make use of the fact that the pattern traces an approximate great circle path. By rotating the global coordinate system such that the equator (itself a great circle path) traces the approximate path of the PSA pattern, our identification algorithm could simply apply a Fourier transform along the equator in the new zonal direction. The algorithm is described in detail below, along with the other more general data analysis techniques used in the study.

\subsection{Identification algorithm}

\subsubsection{Grid rotation}

In order to align the equator with the approximate path of the PSA pattern, a new global 0.75$^{\circ}$ latitude by 0.75$^{\circ}$ longitude grid was defined (i.e. the same resolution as the original ERA-Interim data) with the north pole located at 20$^{\circ}$N, 260$^{\circ}$E. The ERA-Interim zonal and meridional wind data were then used to re-calculate the meridional wind relative to this new north, and this new meridional wind was then linearly interpolated to the new grid (i.e. because the ERA-Interim grid is irregularly spaced relative to the new grid). The anomaly of the new meridional wind was used for PSA pattern identification. It should be noted that existing zonal wave studies \citep[e.g.][]{IrvingSimmonds2015} tend to skip this final step of calculating the anomaly, because the temporal mean of the meridional wind is typically close to zero (and hence waveforms already oscillate back and forth along the search path). An example of wind rotation process is shown in FIGURE 2. 

\subsubsection{Fourier transform}

The search domain is 10S to 10N and 115E to 235E in the rotated world. We average over a domain to smooth out the bumps and because we want a pattern that spans a large domain.

Zero pad: If you're looking for a wave that does not complete an integer number of cycles in the given domain, common practice is to "zero pad" until the number of sample points are at least a single integer multiple of the number of cycles (the more padding, the finer resolution your transform frequencies are - ideally you'd pad to infinity)

When you zero pad, you are essentially multiplying your data by a square wave (value 1 for data you're keeping and 0 elsewhere). This "convolution" of two waves has consequences in the frequency space, such that even a wave that would repeat exactly 6 times (for example) over the zero padded domain will show power at other frequencies (although in the focus frequency it will have the full value). This is called spectral leakage (into the sidelobes) and in this case is due to the fact that a square wave in the space or time dimension isn't square in the frequency dimension.

This leakage into the sidelobes is a problem for some people, so instead of multiplying by a square "window function", they'll pick something like a Hanning or Hamming Window (there are numpy and scipy functions for this). In the frequency space these windows don't have so much spread into the sidelobes, which has the result of producing a periodogram that has less spread into the sidelobes, but the main lobes are flattened out more (i.e. the focus frequency won't have the full value).

\subsubsection{Selection}

