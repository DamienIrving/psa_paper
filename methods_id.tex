\section{Methodology}

Many existing climatologies of Rossby wave activity focus on zonally propagating waves, whereby the wave identification method simply applies a Fourier transform along adjacent lines of constant latitude \citep[e.g.][]{Glatt2014,IrvingSimmonds2015}. More sophisticated methods have been developed for tracking non-zonal synoptic-scale Rossby wave packets \citep[e.g.][]{Zimin2006,Souders2014}, however such sophistication was not needed in devising a method for objectively identifying the PSA pattern. Instead, it was possible to make use of the fact that the pattern traces an approximate great circle path. By rotating the global coordinate system such that the equator (itself a great circle path) traces the approximate path of the PSA pattern, our identification algorithm could simply apply a Fourier transform along the equator in the new zonal direction. This algorithm is described below, along with the other more general data analysis techniques used in the study. The latter share many similarities with the techniques used by \citet{IrvingSimmonds2015}.

\subsection{Identification algorithm}

\subsubsection{Grid rotation}

In order to align the new equator with the approximate path of the PSA pattern, a global 0.75$^{\circ}$ latitude by 0.75$^{\circ}$ longitude grid was defined (i.e. the same resolution as the original ERA-Interim data) with the north pole located at 20$^{\circ}$N, 260$^{\circ}$E. The 500 hPa zonal and meridional wind data were used to re-calculate the meridional wind relative to the new north, and then the temporal anomaly of this new meridional wind was linearly interpolated to the rotated grid for use in the Fourier transform (e.g. Figure \ref{fig:rotation}). It should be noted that existing zonal wave studies \citep[e.g.][]{IrvingSimmonds2015} tend to skip this final step of calculating the anomaly, because in the case of zonal waves the temporal mean of the meridional wind is typically close to zero (and hence waveforms defined by the meridional wind already oscillate about zero). 

On this rotated grid, the search region of interest was defined as the area bounded by 10$^{\circ}$S to 10$^{\circ}$N and 115$^{\circ}$E to 235$^{\circ}$E (this approximate area is referred to as the PSA-sector at times throughout the paper). This region was selected via visual comparison with existing definitions of the PSA pattern (e.g. Figure \ref{fig_eof}), however the final results were not sensitive to small changes in pole location or search region bounds.

\subsubsection{Fourier transform}

To prepare the meridional wind anomaly for Fourier transform, the meridional mean was calculated over 10$^{\circ}$S to 10$^{\circ}$N (in order to eliminate the latitudinal dimension) and then values outside of 115$^{\circ}$E to 235$^{\circ}$E were set to zero. Zero padding is a commonly used technique when the waveform of interest does not complete an integer number of cycles in a given domain, and is equivalent to multiplying the original signal (in this case the meridional mean meridional wind anomaly) by a square window function. This multiplication (or convolution) of two waves has consequences in frequency space, such that even a perfectly sinusoidal signal that would repeat exactly six times (for example) over the zero padded domain would show power at more than one frequency. This phenomenon is known as spectral leakage (into the side lobes of the frequency spectrum) and arises due to the fact that a square window function is not square in frequency space. In analyses where excessive leakage is undesirable, a Hanning or Hamming window can be used instead. In the frequency space these windows do not display as much spread into the side lobes, however this comes at the expense at the magnitude of the main lobes. Since our selection process (see below) focuses identifying the main lobes, a square window function was considered most appropriate.

\subsubsection{Identification and characterization of PSA-like variability}

Given that the PSA pattern completes approximately 1.6 to 2.0 cycles (depending on the specific EOF mode) over the 120$^{\circ}$ search area (see Figure \ref{fig:eof}), our analysis focused data times where the Fourier transform revealed wavenumber 5 and 6 as dominant frequencies over the zero padded 360$^{\circ}$ domain. In particular, a data time was said to display PSA-like variability (and hence was selected for further analysis) if the amplitude of the wavenumber 5 and 6 components of the Fourier transform were ranked in the top three of all frequencies. The vague "PSA-like" descriptor is used because a number of features besides the PSA pattern (e.g. Antarctic Dipole, Amundsen Sea Low, zonal wave three pattern) can exhibit wavenumber 5-6 variability in the PSA-sector.

Once these data times were selected, additional information from the Fourier transform was used to characterize the phase and amplitude of the PSA-like variability. With respect to the former, it can be seen from Figure \ref{fig:transform} that within the search area the phase of the wavenumber 5 and 6 components of the transform (and usually also adjacent frequencies like wavenumber 4 and 7) tend to align both with each other and also with the phase of the actual signal. The phase of the wavenumber 6 component of the Fourier transform was therefore used as a proxy for the phase of the signal as a whole, and this information was used to try and separate data times displaying the PSA pattern from times of general PSA-like variability. In order to quantify the amplitude, the wave envelope construct pioneered (in the atmospheric sciences) by \citet{Zimin2003} and recently applied by \citet{IrvingSimmonds2015} was used. The envelope of complete signal (i.e. with all wavenumbers retained) can be quite noisy, so instead the amplitude was defined as the maximum value of the envelope when only wavenumbers 4 to 7 are retained (see Figure \ref{fig:transform} for an example of the envelope).

\subsubsection{Timescale considerations}

In applying the identification algorithm to the ERA-Interim dataset, the decision was made to focus on monthly timescale data at 500 hPa. The latter represents a mid-to-upper tropospheric level that is below the tropopause in all seasons and at all latitudes of interest. Given the equivalent barotropic nature of the PSA pattern (i.e. the wave amplitude increases with height but phase lines tend to be vertical) the results do not differ substantially for other levels of the troposphere. Monthly mean data were used for consistency with most previous studies and were obtained by applying a 30-day running mean to the daily (i.e. diurnally averaged) ERA-Interim data, in order to maximize the available monthly timescale information. As noted by previous authors \citep[e.g.][]{Kidson1988}, potentially useful information may be lost if only twelve (i.e. calendar month) samples are taken every year. Dates were labeled as the 16th day of the 30-day period (e.g. the labeled date 1979-01-16 spans the period 1979-01-01 to 1979-01-30).

To explore the implications of this timescale selection, the Fourier transform used in the identification process was applied to the 500 hPa rotated meridional wind anomaly data (FIGURE 4). That analysis revealed wavenumber 7 as the most dominant frequency for daily timescale data in the PSA-sector, with wavenumber 6 dominating the frequency spectrum for a 10-90 day running window. Given that the PSA pattern is itself characterized by wavenumber 5-6 variability in the PSA-sector, this result suggests that (a) the PSA pattern is a dominant regional feature on weekly through to seasonal timescales, and (b) by extension the climatological results obtained from 30-day running mean data are also relevant at those timescales.

