\section{Introduction}

The Pacific South-American (PSA) pattern has long been recognized as an important mode of regional climate variability. First named by \citet{Mo1987}, the pattern was identified in a number of studies of the large-scale Southern Hemisphere (SH) circulation during the late 1980s and early 1990s \citep[e.g.][]{Lau1994}. A link between the pattern and Rossby wave dispersion associated with the El Ni\~{n}o Southern Oscillation (ENSO) was soon found \citep[e.g.][]{Karoly1989}, and this work was followed by a number of detailed analyses of the characteristics of the pattern and its downstream impacts \citep[e.g.][]{Mo1998,Mo2000,Mo2001}. In the time since these initial climatological accounts, substantial advances have been made in the methods and datasets used to identify quasi-stationary Rossby wave patterns. Given that the PSA pattern has been implicated in recent Antarctic temperature trends, these advances could be employed to better understand the role of the pattern in high latitude climate variability and its climatological characteristics more generally.

The PSA pattern is most commonly analyzed with respect to a pair of Empirical Orthogonal Function (EOF) patterns (e.g. Figure \ref{fig:eof}). Known as PSA-1 and PSA-2, these patterns are in quadrature and depict a wave train extending along an approximate great circle path from the central Pacific Ocean to the Amundsen and Weddell Seas. Some authors interpret these patterns as a single eastward propagating wave \citep{Mo1998}, while others argue that variability in the PSA sector is better described as a set of geographically fixed regimes \citep{Robertson2003}. On a decadal time scale, PSA-1 has been related to sea surface temperature (SST) anomalies over the central and eastern Pacific, while on an interannual time scale it appears as a response to ENSO \citep{Mo2001}. The association of PSA-2 with tropical variability is less clear, with some authors relating it to the quasi-biennial component of ENSO variability \citep{Mo2000} and others to the Madden Julian Oscillation \citep{Renwick1999}. While most of the features of the PSA pattern are consistent with theory and/or modelling of Rossby wave dispersion from anomalous tropical heat sources \citep[e.g.][]{Liu2007,Li2015}, it is recognized that the pattern can also result from internal atmospheric fluctuations caused by instabilities of the basic state \citep[and that both mechanisms likely act in concert; e.g.][]{Grimm2009}.

It has been shown that the PSA pattern plays a role in blocking events \citep{Sinclair1997,Renwick1999} and South American rainfall variability \citep{Mo2001} and is also closely related to prominent regional features such as the Amundsen Sea Low \citep{Turner2013}, Antartic Dipole \citep{Yuan2001}, Antarctic Circumpolar Wave \citep{Christoph1998} and Southern Annular Mode \citep[SAM; e.g.][]{Ding2012}. While these are all important mid-to-high latitude impacts and relationships, in recent years the PSA pattern has been mentioned most frequently in the literature in relation to the rapid warming observed over West Antarctica and the Antarctic Peninsula \citep{Nicolas2014}. In particular, it has been suggested that seasonal trends in tropical Pacific SSTs may be responsible, via circulation trends resembling the PSA pattern, for winter (and to a lesser extent spring) surface warming in West Antarctica \citep{Ding2011} and autumn surface warming across the Antarctic Peninsula \citep{Ding2013}. The pattern has also been associated with declines in sea ice in the Amunden and Bellingshausen Seas \citep{Schneider2012} and glacier retreat in the Amundsen Sea Embayment \citep{Steig2012}.

In identifying the PSA pattern as a possible contributor to these trends, the aforementioned studies looked through the lens of the variable/s of interest. For instance, \citet{Ding2011} performed a maximum covariance analysis to look at the relationship between central Pacific SSTs and the broader SH circulation (the 200hPa geopotential height). The second mode of that analysis revealed a circulation resembling the PSA pattern (and that brings warm air over West Antarctica), and atmospheric model runs forced with the associated central Pacific SSTs produced a PSA-like wave train. While this is certainly a valid research methodology, the result would be more robust if a climatology of PSA pattern activity also displayed trends consistent with warming in West Antarctica. This concept of teleconnection reversibility was recently invoked to question the relationship between Indian Ocean SSTs and heat waves in south-western Australia \citep{Boschat2016}.  

Given that our current climatological understanding of the PSA pattern is somewhat dated \citep{Mo1998,Mo2001}, this study will seek to present an update. Not only will it utilize a longer, higher quality reanalysis dataset than previous studies, it will also develop and apply a methodology that fully exploits the capabilities of Fourier analysis, as opposed to relying on a traditional EOF-based approach. This alternative methodology was adapted from a recent climatology of SH zonal wave activity \citep{IrvingSimmonds2015} and seeks to avoid the issues associated with the stationary nature of spatial EOF modes, which can be problematic when trying to capture phase variations in a wave pattern of interest. These issues are further compounded in the case of the PSA pattern, due to the degenerate \citep{North1982} nature of the PSA-2 mode \citep[e.g. Figure 1;][]{Mo2000}. This updated climatology will provide new insights into the variability, propagation and downstream impacts of the PSA pattern, including its the role in recent high latitude trends. 

