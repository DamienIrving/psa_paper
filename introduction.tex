\section{Introduction}

The Pacific South-American (PSA) pattern has long been recognized as an important mode of regional climate variability. Formally named by \cite{Mo1987}, the pattern was identified in a number of studies of the large-scale Southern Hemisphere (SH) circulation during the late 1980s and early 1990s \citep[e.g.][]{Lau1994}. A link between the pattern and Rossby wave dispersion associated with El Nino Southern Oscillation (ENSO) was soon found \citep[e.g.][]{Karoly1989}, and this work was followed by a number of detailed analyses of the characteristics of the pattern and its downstream impacts \citep[e.g.][]{Mo1998,Mo2000,Mo2001}. In the time since these initial climatological accounts, substantial advances have been made in the methods and datasets used to identify quasi-stationary Rossy wave patterns. Given that the PSA pattern has been implicated in the recent climatic changes observed over and around West Antarctica, these advances could be employed to better understand the role of the pattern in high latitude climate variability and its climatological characteristics more generally.

The PSA pattern is most commonly analyzed with respect to a pair of empirical orthogonal function (EOF) patterns. Known as PSA-1 and PSA-2, these patterns are in quadrature (i.e. they are 90 degrees out of phase with one another) and depict a wave train extending from the central Pacific south and east over the Amundsen and Weddell Seas via an approximate great circle path.   