\section{Introduction}

The Pacific South-American (PSA) pattern has long been recognized as an important mode of regional climate variability. Formally named by \cite{Mo1987}, the pattern was identified in a number of studies of the large-scale Southern Hemisphere (SH) circulation during the late 1980s and early 1990s \citep[e.g.][]{Lau1994}. A link between the pattern and Rossby wave dispersion associated with El Nino Southern Oscillation (ENSO) was soon found \citep[e.g.][]{Karoly1989}, and this work was followed by a number of detailed analyses of the characteristics of the pattern and its downstream impacts \citep[e.g.][]{Mo1998,Mo2000,Mo2001}. In the time since these initial climatological accounts, substantial advances have been made in the methods and datasets used to identify quasi-stationary Rossy wave patterns \citep[e.g.][]{Irving2015}. Given that the PSA pattern has been implicated in recent climatic changes observed over West Antarctica \citep{Ding2011} and the Antarctic Peninsula \citep{Ding2013}, these advances could be employed to better understand the role of the pattern in high latitude climate variability and its climatological characteristics more generally.

The PSA pattern is most commonly analyzed with respect to a pair of empirical orthogonal function (EOF) patterns. Known as PSA-1 and PSA-2, these patterns are in quadrature and depict a wave train extending along an approximate great circle path from the central Pacific Ocean to the Amundsen and Weddell Seas. Some authors interpret these patterns as a single eastward propagating wave \citep{Mo1998}, while others argue that variability in the PSA sector is better described as a set of geographically fixed regimes \citep{Robertson2003}. On a decadal time scale, the strength of PSA-1 has been related to sea surface temperature (SST) anomalies over the central and eastern Pacific, while on an interannual time scale it appears as a response to ENSO \citep{Mo2001}. The association of PSA-2 with tropical variability is less clear, with some authors relating it to the quasi-biennial component of ENSO variability \citep{Mo2000} and others to the Madden Julian Oscillation \citep{Renwick1999}. While the patterns are largely consistent with Rossby wave dispersion theory, it is also recognised that internal variability plays a role.

