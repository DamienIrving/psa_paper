\section{Introduction}

\begin{itemize}
\item The literature on teleconnections between the tropical Pacific and higher latitudes dates back to the seminal work of Horel & Wallace, Karoly etc. (\citet{Ciasto2015} give a nice summary of the tropical Pacific teleconnection literature in their introduction).
\item Papers from Mo etc then went ahead and looked at climate variability over the South Pacific more broadly and named the PSA pattern (and Antarctic Dipole) and quantified its basic climatological characteristics. Suggested that the pattern isn't always externally forced - internal variability plays a role too.
\item Recent changes in the tropical Pacific (different ENSO flavors) and high southern latitudes (warming in West Antarcitca) have reignited interest in this area. A bunch of people have run models with different SST patterns to see how the location of SST anomalies changes the behaviour of the PSA pattern, while regressions and maximum covariance analysis of high latitude temperature have found a PSA-like pattern. 
\item It is therefore probably a good time to re-visit those initial studies. \citet{Li2015} have looked at the wave theory, and we will look at variability more broadly. As compared to the initial studies, we have a longer time period, better data, and we can build upon recent advances in the identification of stationary wave patterns \citep[e.g.][]{Irving2015}. By looking through the lens of the PSA pattern (as opposed to tropical SSTs or high latitude temperatures) hopefully we can provide new insights.
\end{itemize}
  