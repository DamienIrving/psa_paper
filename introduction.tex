\section{Introduction}

The Pacific South-American (PSA) pattern has long been recognized as an important mode of regional climate variability. Formally named by \cite{Mo1987}, the pattern was identified in a number of analyses of the large-scale Southern Hemisphere circulation during the late 1980s and early 1990s \citep[e.g.][]{Lau1994}. A link between the pattern and Rossby wave dispersion associated with El Nino Southern Oscillation (ENSO) was soon identified \citep[e.g.][]{Karoly1989}, and this work was followed by a number of more detailed analyses of the characteristics of the pattern and its downstream impacts \citep[e.g.][]{Mo1998,Mo2000,Mo2001}. Interest in the PSA pattern has resurfaced in recent years, in relation to the rapid climatic changes observed in and around West Antarctica. Given the global significance of these changes, we feel that it is important to revisit our understanding of the PSA pattern and its role in high latitude climate variability.     
