\section{Introduction}

The Pacific South-American (PSA) pattern has long been recognized as an important mode of regional climate variability. Formally named by \cite{Mo1987}, the pattern was identified in a number of studies of the large-scale Southern Hemisphere (SH) circulation during the late 1980s and early 1990s \citep[e.g.][]{Lau1994}. A link between the pattern and Rossby wave dispersion associated with El Nino Southern Oscillation (ENSO) was soon found \citep[e.g.][]{Karoly1989}, and this work was followed by a number of detailed analyses of the characteristics of the pattern and its downstream impacts \citep[e.g.][]{Mo1998,Mo2000,Mo2001}. In the time since these initial climatological accounts, substantial advances have been made in the methods and datasets used to identify quasi-stationary Rossy wave patterns. Given that the PSA pattern has recently been implicated in the rapid climatic changes observed in and around West Antarctica... these advances could be employed to better understand the role of the PSA pattern in high latitude climate variability and its characteristics more generally.







Given the global significance of these changes, we feel that it is important to revisit our understanding of the PSA pattern and its role in high latitude climate variability.     
