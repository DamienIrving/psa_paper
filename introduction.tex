\section{Introduction}

\begin{itemize}
\item What is the PSA pattern?
\begin{itemize}
\item The literature on teleconnections between the tropical Pacific and higher latitudes dates back to the seminal work of Horel & Wallace, Karoly etc. (\citet{Ciasto2015} give a nice summary of the tropical Pacific teleconnection literature in their introduction).
\item Papers from Mo etc then went ahead and looked at climate variability over the South Pacific more broadly and named the PSA pattern (and Antarctic Dipole) and quantified its basic climatological characteristics. Suggested that the pattern isn't always externally forced - internal variability plays a role too.
\end{itemize}

\item Why is it important?
\begin{itemize}
\item Recent changes in the tropical Pacific (different ENSO flavors) and high southern latitudes (warming in West Antarcitca) have reignited interest in this area. A bunch of people have run models with different SST patterns to see how the location of SST anomalies changes the behaviour of the PSA pattern, while regressions and maximum covariance analysis of high latitude temperature have found a PSA-like pattern. 
\end{itemize}

\item What are the things we don't know about? 
\begin{itemize}
\item (Focus on the area this paper contributes to)
\item Spatial structure (are PSA-1 and PSA-2 both a thing? Is there anything in between? How does it transition between those modes (i.e. is it stationary or propagating)
\item Are there any trends and seasonality in the counts? (One paper suggests tropics and related to West Antarctic variability but the trends are related to the Atlantic)
\item Relationships: tropics (SSTs including ENSO, Rossby Wave Source), SAM (there is some PSA in the SAM pattern; the phase of the SAM influences the atmospheric bridge), Antarctic Dipole, ZW3
\item Impacts (temperature, precipitation, sea ice) 
\item Whether the pattern moves E/W is probably not something I can look at (the moves are so subtle and my method is not that sensitive)
\end{itemize}

\item What will this study do?
\begin{itemize}
\item It is therefore probably a good time to re-visit those initial studies. \citet{Li2015} have looked at the wave theory, and we will look at variability more broadly. As compared to the initial studies, we have a longer time period, better data, and we can build upon recent advances in the identification of stationary wave patterns \citep[e.g.][]{Irving2015}. By looking through the lens of the PSA pattern (as opposed to tropical SSTs or high latitude temperatures) hopefully we can provide new insights.
\item i.e. what new insights can be obtained on these topics by looking from a PSA-centric viewpoint and by adapting some recent advances in the identification of stationary wave patterns?
\end{itemize}
\end{itemize}
  